\documentclass[
    paper=a4,
    % fontsize=12,
    % DIV=calc,
]{scrartcl}

\usepackage[T1]{fontenc}	
\usepackage[utf8]{inputenc}
\usepackage[ngerman]{babel}
\usepackage{microtype}
\usepackage{libertine}
\usepackage[libertine]{newtxmath} 

\usepackage{xcolor}
\usepackage{booktabs}
\usepackage{graphicx}
\usepackage{subcaption}
\usepackage{pdfpages}
\usepackage{svg}
\usepackage[
    locale		=	DE, 					%	Deutsche Normen
	detect-all,								%	Richtige Font im Textmodus/Mathemodus
	per-mode	=	fraction,				%	Bruchstrich darstellen
	range-units	=	repeat,					%	Einheiten wiederholt darstellen bei sirange
	range-phrase=	{{ bis }},	
]{siunitx}
\usepackage{amsmath}
\usepackage{circuitikz}


\usepackage{hyperref}

\definecolor{colorblue}{HTML}{0480CC}				% color for weblinks
\definecolor{colordarkblue}{HTML}{4C1DCC}			% no use case at the moment
\definecolor{colorgreen}{HTML}{26CC1B}				% color for missing macro
\definecolor{colorred}{HTML}{CC1204}				% color for edit macro


\KOMAoptions{
    numbers=noendperiod,
}

\hypersetup{
    bookmarksnumbered   =   true,
    breaklinks          =   true,
    colorlinks          =   true,
    linkcolor           =   colorred,
    urlcolor            =   colorblue,
    pdftitle            =   {Die digitale Zweifaltigkeit},
    pdfsubject          =   {Versuchsbericht Labor Digitaltechnik},    
    pdfauthor           =   {Jan Hoegen},
}

\ctikzset{
    logic ports=european,
    logic ports/scale=1.0,
    logic ports/fill=lightgray,
    tripoles/european not symbol=ieee circle,
}
\usetikzlibrary{babel}

\graphicspath{/Anhang}

\newcommand{\shadowsection}[1]{%
	\refstepcounter{section}
	\addcontentsline{toc}{section}{\protect\numberline{\thesection}{#1}}
}

\newcommand{\legend}[1]{\par\footnotesize\textbf{Legende}: #1\par}
\newcommand{\figsource}[1]{\par\footnotesize\textbf{Quelle:} #1\par}

\newcommand{\quoteenv}[1]{\glqq #1\grqq} 

\titlehead{
    \textsc{Hochschule Karlsruhe}\\
    University of Applied Sciences\\
    Studiengang EITB SS 22}
\subject{Digitaltechnik Labor 2}
\title{Von halben und vollen Addierern}
\subtitle{Vorbereitung}
\author{Jan Hoegen\thanks{Matrikelnumer: 82358}}
\publishers{Betreuer: Prof. Dr.\,-Ing. Jan Bauer}
\date{Erstellt am: \today}


\begin{document}

\maketitle

\tableofcontents

\newpage

\section{Umformung des Halbaddierers}
    In diesem Abschnitt wird ein Halbaddierer durch ausschließlich NAND- und NOR-Gatter realisiert. Nachdem im nächsten Abschnitt auf gleiche Weise Ein Volladdierer umgeformt wird, kann schließlich ein 2-Bit-Ripple-Carry-Addierer nur durch NANDs und NOrs gebildet werden.

    \subsection{Umformen der Gleichung}
        Für den \textit{Carry-Out} des Halbaddieres gilt gemäß der Laboranleitung:
        \[c_0 = a \wedge b\]

        Nach Umformen durch Idempotenz und Involution ergibt sich:
        \begin{align*}
            c_0 &= a \wedge b\\
            c_0 &= ( a \wedge b ) \wedge ( a \wedge b )\\
            c_0 &= \overline{\overline{( a \wedge b ) \wedge ( a \wedge b )}}\\
            c_0 &= \overline{\overline{(a \wedge b)} \wedge \overline{(a \wedge b)}}\\
        \end{align*}

        Dies entspricht 2 NAND-Gattern, der Ausgang des Ersten liegt dabei auf beiden Eingängen des Zweiten.

    \subsection{Schaltbild des umgeformten Halbaddierers}
        \label{subsec:1}
        Der vollständige Halbaddierer mit Umformung nach oben stehender Rechnung ergibt folgendes Schaltbild:
        \begin{center}
            \begin{tikzpicture}
                \node(a) at (0,0.23) {$a$};
                \node(b) at (0,-0.23) {$b$};
                \node(s) at (8,0){$s$};
                \node(c) at (8,-1.5) {$c_0$};
                \node[xor port] (XOR2) at (3,0){};    
                \node[nand port] (NANDa) at (3,-1.5){};
                \node[nand port] (NANDc) at (6,-1.5){};

                \draw (XOR2.out) |- (s);
                \draw (NANDc.out) -- (c);
                
                \draw (a) -| (XOR2.in 1);
                \draw (0.5,0.23) node[circ]{} |-  (NANDa.in 2);

                \draw (b) -| (XOR2.in 2) node[circ]{} |-  (NANDa.in 1);

                \draw (NANDa.out) -| node[circ, midway]{} (NANDc.in 1) |- (NANDc.in 2);
            \end{tikzpicture}
        \end{center}

    \subsection{Simulation der Schaltung}
       Die oben stehende Schaltung in logisim-evolution aufgebaut erzeugt die Wahrheitstabelle aus \ref{tab:1}. Sie stimmt mit der eines Halbaddierers überein, die Schaltung erfüllt damit weiterhin ihre Funktion.
        \begin{table}[ht]
            \centering
            \caption{Wahrheitstabelle des simulierten Halbaddierers}
            \label{tab:1}
            \begin{tabular}{cc|cc}\toprule
                $a$&$b$&$c$&$s$\\\midrule
                $0$&$0$&$0$&$0$\\
                $0$&$1$&$0$&$1$\\
                $1$&$0$&$0$&$1$\\
                $1$&$1$&$1$&$0$\\\bottomrule
                \end{tabular}
        \end{table}

% \clearpage

\section{Umformen des Volladdierers}
    Nachdem zuerst der Halbaddierer durch NORs und NANDs realisiert wurde, wird nun analog mit dem Volladdierer vorgegangen.

    \subsection{Umformen der Gleichung}

        Für den Ausgang des \textit{Carry-Out} eines Volladdierers gilt gemäß der Laboranleitung: 
        \[c_0 = (a \wedge b) \vee (\bar{s} \wedge c_i)\]

        Nach Umformen durch Involution ergibt sich:
        \begin{align*}
            c_0 &= (a \wedge b) \vee (\bar{s} \wedge c_i)\\
            c_0 &= \overline{\overline{(a \wedge b) \vee (\bar{s} \wedge c_i)}}\\
            c_0 &= \overline{(\overline{a \wedge b}) \wedge (\overline{\bar{s} \wedge c_i}})
        \end{align*}

        Dies entspricht drei NAND-Gattern. Jede Klammer sowie ihre Verknüpfung ist ein NAND-Gatter. 

    \subsection{Schaltbild des umgeformten Volladdierers}
        \label{subsec:2}
        Das vollständige Schaltbild eines Volladdierers nach oben stehender Umformung sieht dann so aus:

        \begin{center}
            \begin{tikzpicture}
                \node (a) at (0,0.23) {$a$};
                \node (b) at (0,-0.23) {$b$};
                \node (c) at (0,-1.73) {$c_i$};
                \node (s) at (11,-0.23) {$s$};
                \node (cb) at (11, -1.73) {$c_0$};
                \node[xor port] (XOR2) at (3,0){};    
                \node[xor port] (XORb) at (9,-0.23){};
                \node[nand port] (NANDa) at (6,-1.5){};  
                \node[nand port] (NANDb) at (9,-1.73){};    
                \node[nand port] (NANDc) at (3,-3){};     

                \draw (a) -| (XOR2.in 1);
                \draw (0.5,0.23) node[circ]{} |-  (NANDc.in 2);

                \draw (b) -| (XOR2.in 2) node[circ]{} |- (NANDc.in 1);

                \draw (XOR2.out) -| node[circ]{} (NANDa.in 1) |- (XORb.in 1);

                \draw (c) -| (NANDa.in 2);
                \draw (3,-1.73) node[circ]{} |- (XORb.in 2);

                \draw (XORb.out) |- (s);
                \draw (NANDa.out) -| (NANDb.in 1);
                \draw (NANDc.out) -| (NANDb.in 2);
                \draw (NANDb.out) |- (cb);
            \end{tikzpicture}
        \end{center}

    \subsection{Simulation der Schaltung}
        Diese Schaltung in logisim-evolution aufgebaut erzeugt die Wahrheitstabelle in \ref{tab:2}. Sie stimmt mit der eines Volladdierers überein, die Schaltung erfüllt also weiterhin ihre Funktion.

        \begin{table}[h]
            \centering
            \caption{Wahrheitstabelle des simulierten Volladdierers}
            \label{tab:2}
            \begin{tabular}{ccc|cc}\toprule
                $a$&$b$&$c_i$&$c_0$&$s$\\\midrule
                $0$&$0$&$0$&$0$&$0$\\
                $0$&$0$&$1$&$0$&$1$\\
                $0$&$1$&$0$&$0$&$1$\\
                $0$&$1$&$1$&$1$&$0$\\
                $1$&$0$&$0$&$0$&$1$\\
                $1$&$0$&$1$&$1$&$0$\\
                $1$&$1$&$0$&$1$&$0$\\
                $1$&$1$&$1$&$1$&$1$\\\bottomrule
            \end{tabular}
        \end{table}

% \clearpage

\section{Umformen des 2-Bit-Addierer-Netz}

    \subsection{Blockschaltbild des 2-Bit Ripple-Carry-Addiernetz}
        Das Schaltbild des 2-Bit Ripple-Carry-Addiernetz in Blockschaltung sieht so aus:

        \begin{center}
            \begin{tikzpicture}
                \node (a0) at (0,-0.15) {$a_0$};
                \node (b0) at (0,-1.85) {$b_0$};
                \node (a1) at (0,-4.15) {$a_1$};
                \node (b1) at (0,-5.85) {$b_1$};
                \node (s0) at (8,-0.15) {$s_0$};
                \node (s1) at (8,-4.15) {$s_1$};
                \node (c1) at (8,-5.85) {$c_1$};
                
                \node [flipflop] (ha) at (3,-1){HA};
                \node [flipflop] (va) at (3,-5){VA};

                \draw (a0) -- (ha.bpin 1);
                \draw (b0) -- (ha.bpin 3);
                \draw (ha.bpin 6) -- (s0);
                \draw (ha.bpin 4) -| (5,-3) node[right]{$c_0$} -| (va.bup);

                \draw (a1) -- (va.bpin 1);
                \draw (b1) -- (va.bpin 3);
                \draw (va.bpin 6) |- (s1);
                \draw (va.bpin 4) |- (c1);
            \end{tikzpicture}
        \end{center}

    \subsection{Umformen des Schaltbilds}
        Das vollständige Schaltbild mit einsetzen der Layouts aus Abschnitt \ref{subsec:1} und \ref{subsec:2} ergibt: 

        \begin{tikzpicture}
            \node (a0) at (0,0.23) {$a_0$};
            \node (b0) at (0,-0.23) {$b_0$};
            \node (a1) at (0,-2.77) {$a_1$};
            \node (b1) at (0,-3.23) {$b_1$};
            \node (s0) at (14,0) {$s_0$};
            \node (s1) at (14,-1.73) {$s_1$};
            \node (c1) at (14,-3.48) {$c_1$};

            \node [xor port] (XOR1) at (3,0) {};
            \node [xor port] (XOR2) at (9,-1.73) {};
            \node [xor port] (XOR3) at (3,-3) {};
            
            \node [nand port] (NAND1) at (3,-1.5) {};
            \node [nand port] (NAND2) at (6,-1.5) {};
            \node [nand port] (NAND3) at (9,-3.25) {};
            \node [nand port] (NAND4) at (3,-4.5) {};
            \node [nand port ](NAND5) at (12,-3.48) {};

            \draw (a0) -| (XOR1.in 1);
            \draw (0.5,0.23) node[circ]{} |- (NAND1.in 2);
            
            \draw (b0) -| (XOR1.in 2);
            \draw (XOR1.in 2) node[circ]{} -| (NAND1.in 1);

            \draw (XOR1.out) |- (s0);
        
            \draw (NAND1.out) -| node[circ]{} (NAND2.in 1) -| (NAND2.in 2);

            \draw (a1) -| (XOR3.in 1);
            \draw (0.5,-2.77) node[circ]{} |- (NAND4.in 2);

            \draw (b1) -| (XOR3.in 2);
            \draw (XOR3.in 2) node[circ]{} -| (NAND4.in 1);

            \draw (NAND2.out) -| (XOR2.in 1);

            \draw (XOR3.out) -| node[circ]{} (XOR2.in 2) -| (NAND3.in 1);
            \draw (XOR2.out) |- (s1);

            \draw (6.5,-1.5) node[above]{$c_0$} node[circ]{} |- (NAND3.in 2);

            \draw (NAND4.out) -| (NAND5.in 2);
            \draw (NAND3.out) -| (NAND5.in 1);

            \draw (NAND5.out) |- (c1);
        \end{tikzpicture}

    \subsection{Simulation der Schaltung}
        Diese Schaltung in logisim-evolution aufgebaut erzeugt die Wahrheitstabelle in \ref{tab:3}. Sie stimmt mit der eines 2-Bit-Ripple-Carry-Addierers überein, die Schaltung erfüllt also weiterhin ihre Funktion.

        \begin{table}[h]
            \centering
            \caption{Wahrheitstabelle des simulierten 2-Bit-Ripple-Carry-Addierer}
            \label{tab:3}
            \begin{tabular}{cccc|ccc}\toprule
                $a_0$&$b_0$&$a_1$&$b_1$&$c_1$&$s_1$&$s_0$\\\toprule
                $0$&$0$&$0$&$0$&$0$&$0$&$0$\\
                $0$&$0$&$0$&$1$&$0$&$1$&$0$\\
                $0$&$0$&$1$&$0$&$0$&$1$&$0$\\
                $0$&$0$&$1$&$1$&$1$&$0$&$0$\\
                $0$&$1$&$0$&$0$&$0$&$0$&$1$\\
                $0$&$1$&$0$&$1$&$0$&$1$&$1$\\
                $0$&$1$&$1$&$0$&$0$&$1$&$1$\\
                $0$&$1$&$1$&$1$&$1$&$0$&$1$\\
                $1$&$0$&$0$&$0$&$0$&$0$&$1$\\
                $1$&$0$&$0$&$1$&$0$&$1$&$1$\\
                $1$&$0$&$1$&$0$&$0$&$1$&$1$\\
                $1$&$0$&$1$&$1$&$1$&$0$&$1$\\
                $1$&$1$&$0$&$0$&$0$&$1$&$0$\\
                $1$&$1$&$0$&$1$&$1$&$0$&$0$\\
                $1$&$1$&$1$&$0$&$1$&$0$&$0$\\
                $1$&$1$&$1$&$1$&$1$&$1$&$0$\\\bottomrule    
                \end{tabular}
            \end{table}

\end{document}